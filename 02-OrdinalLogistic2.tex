\end{frame}
%====================================================== %
\begin{frame}
We can also get confidence intervals for the parameter estimates. These can be obtained either by profiling the likelihood function or by using the standard errors and assuming a normal distribution. Note that profiled CIs are not symmetric (although they are usually close to symmetric). If the 95% CI does not cross 0, the parameter estimate is statistically significant.
\end{frame}
%====================================================== %
\begin{frame}
(ci <- confint(m)) # default method gives profiled CIs
## Waiting for profiling to be done...
##          2.5 % 97.5 %
## pared   0.5282 1.5722
## public -0.6522 0.5191
## gpa     0.1076 1.1309
confint.default(m) # CIs assuming normality
##          2.5 % 97.5 %
## pared   0.5268  1.569
## public -0.6426  0.525
## gpa     0.1051  1.127
\end{frame}
%====================================================== %
\begin{frame}
The CIs for both pared and gpa do not include 0; public does. The estimates in the output are given in units of ordered logits, or ordered log odds. So for pared, we would say that for a one unit increase in pared (i.e., going from 0 to 1), we expect a 1.05 increase in the expect value of apply on the log odds scale, given all of the other variables in the model are held constant. 
\end{frame}
%====================================================== %
\begin{frame}
For gpa, we would say that for a one unit increase in gpa, we would expect a 0.62 increase in the expected value of apply in the log odds scale, given that all of the other variables in the model are held constant.
\end{frame}
%====================================================== %
\begin{frame}
The coefficients from the model can be somewhat difficult to interpret because they are scaled in terms of logs. Another way to interpret logistic regression models is to convert the coefficients into odds ratios. To get the OR and confidence intervals, we just exponentiate the estimates and confidence intervals.
\end{frame}
%====================================================== %
\begin{frame}
## odds ratios
exp(coef(m))
##  pared public    gpa 
## 2.8511 0.9429 1.8514
## OR and CI
exp(cbind(OR = coef(m), ci))
##            OR  2.5 % 97.5 %
## pared  2.8511 1.6958  4.817
## public 0.9429 0.5209  1.681
## gpa    1.8514 1.1136  3.098
\end{frame}
%====================================================== %
\begin{frame}
These coefficients are called proportional odds ratios and we would interpret these pretty much as we would odds ratios from a binary logistic regression. For pared, we would say that for a one unit increase in parental education, i.e., going from 0 (Low) to 1 (High), the odds of "very likely" applying versus "somewhat likely" or "unlikely" applying combined are 2.85 greater, given that all of the other variables in the model are held constant. 
\end{frame}
%====================================================== %
\begin{frame}
Likewise, the odds "very likely" or "somewhat likely" applying versus "unlikely" applying is 2.85 times greater, given that all of the other variables in the model are held constant. For gpa (and other continuous variables), the interpretation is that when a student's gpa moves 1 unit, the odds of moving from "unlikely" applying to "somewhat likely" or "very likley" applying (or from the lower and middle categories to the high category) are multiplied by 1.85.
\end{frame}
\end{document}
