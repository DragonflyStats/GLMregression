
\documentclass[a4paper,12pt]{article}
%%%%%%%%%%%%%%%%%%%%%%%%%%%%%%%%%%%%%%%%%%%%%%%%%%%%%%%%%%%%%%%%%%%%%%%%%%%%%%%%%%%%%%%%%%%%%%%%%%%%%%%%%%%%%%%%%%%%%%%%%%%%%%%%%%%%%%%%%%%%%%%%%%%%%%%%%%%%%%%%%%%%%%%%%%%%%%%%%%%%%%%%%%%%%%%%%%%%%%%%%%%%%%%%%%%%%%%%%%%%%%%%%%%%%%%%%%%%%%%%%%%%%%%%%%%%
\usepackage{eurosym}
\usepackage{vmargin}
\usepackage{amsmath}
\usepackage{graphics}
\usepackage{epsfig}
\usepackage{framed}
\usepackage{subfigure}
\usepackage{fancyhdr}

\setcounter{MaxMatrixCols}{10}
%TCIDATA{OutputFilter=LATEX.DLL}
%TCIDATA{Version=5.00.0.2570}
%TCIDATA{<META NAME="SaveForMode"CONTENT="1">}
%TCIDATA{LastRevised=Wednesday, February 23, 201113:24:34}
%TCIDATA{<META NAME="GraphicsSave" CONTENT="32">}
%TCIDATA{Language=American English}

\pagestyle{fancy}
\setmarginsrb{20mm}{0mm}{20mm}{25mm}{12mm}{11mm}{0mm}{11mm}
\lhead{R Programming} \rhead{Statistical Modelling} \chead{Logistic Regression} %\input{tcilatex}


\begin{document}
\section*{South Africa Heart Disease Data Example}

\begin{framed}
\noindent A retrospective sample of males in a heart-disease high-risk region
	of the Western Cape, South Africa. There are roughly two controls per
	case of CHD. Many of the CHD positive men have undergone blood
	pressure reduction treatment and other programs to reduce their risk
	factors after their CHD event. In some cases the measurements were
	made after these treatments.\\ \textit{ These data are taken from a larger
	dataset, described in  Rousseauw et al, 1983, South African Medical
	Journal. }
	
\end{framed}
%Load the South Africa Heart Disease Data and create training and test sets with
%the following code:
%\begin{framed}
%	\begin{verbatim}
%	install.packages("ElemStatLearn")
%	library(ElemStatLearn)
%	data(SAheart)
%	
%	set.seed(8484)
%	train = sample(1:dim(SAheart)[1],
%	size=dim(SAheart)[1]/2,replace=F)
%	trainSA = SAheart[train,]
%	testSA = SAheart[-train,]
%	
%	\end{verbatim}
%\end{framed}

\noindent \textbf{Exercise}\\Fit a logistic regression model with
\begin{itemize}
	\item \textit{Coronary Heart Disease} (\texttt{chd}) as the
	dependent variable
	
	\item \textit{age at onset, current alcohol consumption, obesity levels,
		cumulative tabacco, type-A behavior}, and \textit{low density lipoprotein cholesterol} as predictor variables. 
\end{itemize} 
{
	\large
	
	\begin{verbatim}
	> head(SAheart)
	sbp tobacco  ldl adiposity famhist typea obesity alcohol age chd
	1 160   12.00 5.73     23.11 Present    49   25.30   97.20  52   1
	2 144    0.01 4.41     28.61  Absent    55   28.87    2.06  63   1
	3 118    0.08 3.48     32.28 Present    52   29.14    3.81  46   0
	4 170    7.50 6.41     38.03 Present    51   31.99   24.26  58   1
	5 134   13.60 3.50     27.78 Present    60   25.99   57.34  49   1
	6 132    6.20 6.47     36.21 Present    62   30.77   14.14  45   0
	...
	...
	\end{verbatim}
	
}
\newpage
\noindent Calculate the misclassification rate for your model using this model function and a prediction on the "response" scale:

%\noindent What is the misclassification rate on the training set? What is the misclassification rate on the test set?
\begin{framed}
	\begin{verbatim}
	head(SAheart)
	
	lr1 <- glm(chd ~ age + alcohol + obesity + 
	tobacco + typea + ldl, data=trainSA, 
	family="binomial")
	
	lr1.train.predict <- predict(lr1, type="response")
	
	missclass.lr1.train <- missClass(trainSA$chd, 
	lr1.train.predict)
	
	lr1.test.predict <- predict(lr1, newdata=testSA, 
	type="response")
	
	missclass.lr1.test <- missClass(testSA$chd, 
	lr1.test.predict)
	\end{verbatim}
\end{framed}


\end{document}
