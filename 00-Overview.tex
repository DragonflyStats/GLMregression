\documentclass[00-GLMregslides.tex]{subfiles}
\begin{document}
%==================================================================%
\begin{frame}
\Large
\begin{itemize}
\item Binary logistic regression, also called a logit model, is used to model dichotomous outcome variables. In the logit model the log odds of the outcome is modeled as a linear combination of the predictor variables.
\item Multinomial logistic regression is used to model nominal outcome variables. Again the log odds of the outcomes are modeled as a linear combination of the predictor variables.
\item Ordinal logistic regression is used to model nominal outcome variables, where a hierarchy of categories exists.
\end{itemize}
\end{frame}
%==================================================================%
\begin{frame}
\Large
\begin{itemize}
\item  Poisson regression is used to model count variables.
\item  Negative binomial regression is for modeling count variables, usually for over-dispersed count outcome variables.
\end{itemize}
\end{frame}
%==================================================================%
\begin{frame}[fragile]
\frametitle{Multinomial Logistic Regression with \texttt{R} }
\begin{framed}
\begin{verbatim}


## melt data set to long for ggplot2
lpp <- melt(pp.write, id.vars = c("ses", "write"), value.name = "probability")
head(lpp)  # view first few rows
 
##   ses write variable probability
## 1 low    30 academic     0.09844
## 2 low    31 academic     0.10717
## 3 low    32 academic     0.11650
## 4 low    33 academic     0.12646
## 5 low    34 academic     0.13705
## 6 low    35 academic     0.14828
 



\end{verbatim}
\end{framed}
\end{frame}
%==================================================================%

\begin{frame}[fragile]
\frametitle{Multinomial Logistic Regression with \texttt{R} }
\begin{framed}
\begin{verbatim}

## plot predicted probabilities across write values for each level of ses
## facetted by program type
ggplot(lpp, aes(x = write, y = probability, colour = ses)) + geom_line() + facet_grid(variable ~
    ., scales = "free")

\end{verbatim}
\end{framed}
\end{frame}
%==================================================================%
\begin{frame}[fragile]
\frametitle{Ordinal Logistic Regression with \texttt{R} }
\begin{framed}
\begin{verbatim}

ggplot(dat, aes(x = apply, y = gpa)) +
  geom_boxplot(size = .75) +
  geom_jitter(alpha = .5) +
  facet_grid(pared ~ public, margins = TRUE) +
  theme(axis.text.x = element_text(angle = 45, hjust = 1, vjust = 1))

\end{verbatim}
\end{framed}
\end{frame}
%==================================================================%
\begin{frame}[fragile]
\frametitle{Poisson Regression with \texttt{R} }
\begin{framed}
\begin{verbatim}

with(p, tapply(num_awards, prog, function(x) {
  sprintf("M (SD) = %1.2f (%1.2f)", mean(x), sd(x))
}))
 



\end{verbatim}
\end{framed}
\end{frame}
%==================================================================%

\begin{frame}[fragile]
\frametitle{Poisson Regression with \texttt{R} }
\begin{framed}
\begin{verbatim}

##                General               Academic             Vocational 
## "M (SD) = 0.20 (0.40)" "M (SD) = 1.00 (1.28)" "M (SD) = 0.24 (0.52)"
 
ggplot(p, aes(num_awards, fill = prog)) +
  geom_histogram(binwidth=.5, position="dodge")

\end{verbatim}
\end{framed}
\end{frame}
%==================================================================%
\begin{frame}[fragile]
\frametitle{Negative Binomial Regression with \texttt{R} }
\begin{framed}
\begin{verbatim}


ggplot(dat, aes(daysabs, fill = prog)) + geom_histogram(binwidth = 1) + facet_grid(prog ~ 
    ., margins = TRUE, scales = "free")

\end{verbatim}
\end{framed}
\end{frame}
%==================================================================%
\begin{frame}[fragile]
\begin{framed}
\begin{verbatim}

with(dat, tapply(daysabs, prog, function(x) {
    sprintf("M (SD) = %1.2f (%1.2f)", mean(x), sd(x))
}))

##                 General                Academic              Vocational 
## "M (SD) = 10.65 (8.20)"  "M (SD) = 6.93 (7.45)"  "M (SD) = 2.67 (3.73)"


\end{verbatim}
\end{framed}
\end{frame}
%==================================================================%
\end{document}







