

%================================================================================================%
\documentclass[00-GLMregslides.tex]{subfiles}

\begin{document}

	

%================================================================================================%

\begin{frame}[fragile]

\frametitle{Multinomial Regression with \texttt{R}}
\Large
\begin{itemize}
\item The ratio of the probability of choosing one outcome category over the probability of choosing the baseline category is often referred as 
relative risk (and it is also sometimes referred as odds as we have just used to described the regression parameters above). 
\item  The relative risk is the right-hand side linear equation exponentiated, leading to the fact that the 
exponentiated regression coefficients are relative risk ratios for a unit change in the predictor variable. 
\item  We can exponentiate the coefficients from our model to see these risk ratios. 
\end{itemize}
\end{frame}
%================================================================================================%
\begin{frame}[fragile]

\frametitle{Multinomial Regression with \texttt{R}}
\Large

extract the coefficients from the model and exponentiate
\begin{framed}
\begin{verbatim}
exp(coef(test))
 
          (Intercept) sesmiddle seshigh  write
 general        17.33    0.5867  0.3126 0.9437
 vocation      184.61    1.3383  0.3743 0.8926
\end{verbatim}
\end{framed}
\end{frame}
%================================================================================================%
\begin{frame}[fragile]

\frametitle{Multinomial Regression with \texttt{R}}
\Large
\begin{itemize}
\item The relative risk ratio for a one-unit increase in the variable write is .9437 for being in general program vs. academic program. 
\item The relative risk ratio switching from ses = 1 to 3 is .3126 for being in general program vs. academic program. 
\end{itemize}
\end{frame}
%================================================================================================%
\begin{frame}[fragile]

\frametitle{Multinomial Regression with \texttt{R}}
\Large
\begin{itemize}
\item You can also use predicted probabilities to help you understand the model. 
\item You can calculate predicted probabilities for each of our outcome levels using the fitted function. 
\item We can start by generating the predicted probabilities for the observations in our dataset and viewing the first few rows
\end{itemize}
\end{frame}
%================================================================================================%
\begin{frame}[fragile]

\frametitle{Multinomial Regression with \texttt{R}}
\Large
\begin{verbatim}
head(pp <- fitted(test))
 
   academic general vocation
 1   0.1483  0.3382   0.5135
 2   0.1202  0.1806   0.6992
 3   0.4187  0.2368   0.3445
 4   0.1727  0.3508   0.4765
 5   0.1001  0.1689   0.7309
 6   0.3534  0.2378   0.4088
\end{verbatim} 
Next, if we want to examine the changes in predicted probability associated with one of our two variables, we can create small datasets varying one 
variable while holding the other constant. We will first do this holding write at its mean and examining the predicted probabilities for each level of ses.
 
\end{frame}
%================================================================================================%
\begin{frame}[fragile]

\frametitle{Multinomial Regression with \texttt{R}}
\Large
\begin{verbatim}
dses <- data.frame(ses = c("low", "middle", "high"), write = mean(ml$write))
predict(test, newdata = dses, "probs")
 
   academic general vocation
 1   0.4397  0.3582   0.2021
 2   0.4777  0.2283   0.2939
 3   0.7009  0.1785   0.1206
\end{verbatim}
\end{frame}
%================================================================================================%
\begin{frame}[fragile]

\frametitle{Multinomial Regression with \texttt{R}}
\Large
Another way to understand the model using the predicted probabilities is to look at the averaged predicted probabilities for different values 
of the continuous predictor variable write within each level of ses.
 
\end{frame}
%================================================================================================%
\begin{frame}[fragile]

\frametitle{Multinomial Regression with \texttt{R}}
\Large
\begin{verbatim}
dwrite <- data.frame(ses = rep(c("low", "middle", "high"), each = 41), write = rep(c(30:70),
    3))

 store the predicted probabilities for each value of ses and write
pp.write <- cbind(dwrite, predict(test, newdata = dwrite, type = "probs", se = TRUE))
\end{verbatim}
\end{frame}
%================================================================================================%
\begin{frame}[fragile]

\frametitle{Multinomial Regression with \texttt{R}}
\Large
 calculate the mean probabilities within each level of ses
\begin{verbatim}
by(pp.write[, 3:5], pp.write$ses, colMeans)
 
 pp.write$ses: high
 academic  general vocation 
   0.6164   0.1808   0.2028 
 -------------------------------------------------------- 
 pp.write$ses: low
 academic  general vocation 
   0.3973   0.3278   0.2749 
 -------------------------------------------------------- 
 pp.write$ses: middle
 academic  general vocation 
   0.4256   0.2011   0.3733
\end{verbatim}
\end{frame}
%================================================================================================%
\begin{frame}[fragile]

\frametitle{Multinomial Regression with \texttt{R}}
\Large
\begin{itemize}
\item Sometimes, a couple of plots can convey a good deal amount of information. 
\item Using the predictions we generated for the pp.write object above, we can plot the predicted probabilities against the writing score by the level 
of ses for different levels of the outcome variable.
\end{itemize} 
\end{frame}
%================================================================================================%
\begin{frame}[fragile]

\frametitle{Multinomial Regression with \texttt{R}}
\Large
 melt data set to long for ggplot2
\begin{verbatim}
lpp <- melt(pp.write, id.vars = c("ses", "write"), value.name = "probability")
head(lpp)  # view first few rows
 
   ses write variable probability
 1 low    30 academic     0.09844
 2 low    31 academic     0.10717
 3 low    32 academic     0.11650
 4 low    33 academic     0.12646
 5 low    34 academic     0.13705
 6 low    35 academic     0.14828
\end{verbatim}
\end{frame}
%================================================================================================%
\begin{frame}[fragile]

\frametitle{Multinomial Regression with \texttt{R}}
\Large 
 plot predicted probabilities across write values for each level of ses
 facetted by program type
\begin{verbatim}
ggplot(lpp, aes(x = write, y = probability, colour = ses)) + geom_line() + facet_grid(variable ~
    ., scales = "free")
 
\end{verbatim}
\end{frame}
%================================================================================================%
\begin{frame}[fragile]

\frametitle{Multinomial Regression with \texttt{R}}
\Large
\textbf{Things to consider}
\begin{itemize}
\item The Independence of Irrelevant Alternatives (IIA) assumption: Roughly, the IIA assumption means that adding or deleting alternative outcome categories does not affect the odds among the remaining outcomes. There are alternative modeling methods, such as alternative-specific multinomial probit model, or nested logit model to relax the IIA assumption.
\item Diagnostics and model fit: Unlike logistic regression where there are many statistics for performing model diagnostics, it is not as straightforward to do diagnostics with multinomial logistic regression models. For the purpose of detecting outliers or influential data points, one can run separate logit models and use the diagnostics tools on each model. 
\item Sample size: Multinomial regression uses a maximum likelihood estimation method, it requires a large sample size. It also uses multiple equations. This implies that it requires an even larger sample size than ordinal or binary logistic regression.
\end{itemize}

\end{frame}
%================================================================================================%
\begin{frame}[fragile]

\frametitle{Multinomial Regression with \texttt{R}}
\Large
\textbf{Things to consider}
\begin{itemize}
\item Complete or quasi-complete separation: Complete separation means that the outcome variable separate a predictor variable completely, leading perfect prediction by the predictor variable.
\item Perfect prediction means that only one value of a predictor variable is associated with only one value of the response variable. But you can tell from the output of the regression coefficients that something is wrong. You can then do a two-way tabulation of the outcome variable with the problematic variable to confirm this and then rerun the model without the problematic variable.
\item Empty cells or small cells: You should check for empty or small cells by doing a cross-tabulation between categorical predictors and the outcome variable. If a cell has very few cases (a small cell), the model may become unstable or it might not even run at all.
\end{itemize}

\end{frame}
%================================================================================================%
\end{document}
