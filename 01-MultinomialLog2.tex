

%================================================================================================%

\documentclass[00-GLMregslides.tex]{subfiles}

\begin{document}



%================================================================================================%
\begin{frame}[fragile]
\frametitle{Multinomial Logistic Regression with \texttt{R}}
\Large
\textbf{Multinomial Logistic Regression}
\begin{itemize}
\item 
We use the \texttt{multinom} function from the \textbf{nnet} package to estimate a multinomial logistic regression model. 

\item Remark - There are other functions in other R packages capable of multinomial regression, such as the \textbf{mlogit} package. 
\item
  The \texttt{multinom} function does not require the data to be reshaped (as the \textbf{mlogit} package does) 
\item (Similar format to example code found in Hilbe's \textit{Logistic Regression Models}).
\end{itemize}
\end{frame}
 %================================================================================================%
  \begin{frame}[fragile]
  	\Large
  	\frametitle{Multinomial Logistic Regression with \texttt{R}}
\textbf{Multinomial logistic regression}
\begin{itemize}
 %		\item Before running our model.
\item  We must choose the level of our outcome that we wish to use as our \textbf{\textit{baseline}} and specify this in the \texttt{relevel} function. \\ Let's choose ``academic".
\item Then, we run our model using \textit{multinom}. 
\item The \texttt{multinom} command does not include p-value calculation for the regression coefficients.
\item (We can calculate p-values using Wald tests or z-tests).
 
\end{itemize}
\end{frame}
%================================================================================================%
\begin{frame}[fragile]
\frametitle{Multinomial Logistic Regression with \texttt{R}}
\begin{framed}
	\begin{verbatim}
ml$prog2 <- relevel(ml$prog, ref = "academic")
test <- multinom(prog2 ~ ses + write, data = ml)
 
 # weights:  15 (8 variable)
 initial  value 219.722458 
 iter  10 value 179.982880
 final    value 179.981726 
 converged
\end{verbatim}
\end{framed}
\end{frame}
%================================================================================================%
\begin{frame}[fragile]

\frametitle{Multinomial Logistic Regression with \texttt{R}}
\large

	\begin{verbatim}
summary(test)
 
 Call:
 multinom(formula = prog2 ~ ses + write, 
             data = ml)
 
 Coefficients:
          (Intercept) sesmiddle seshigh    write
 general        2.852   -0.5333 -1.1628 -0.05793
 vocation       5.218    0.2914 -0.9827 -0.11360
\end{verbatim}

\end{frame}

%================================================================================================%
\begin{frame}[fragile]

\frametitle{Multinomial Logistic Regression with \texttt{R}}
\large

\begin{verbatim}
....
 Std. Errors:
          (Intercept) sesmiddle seshigh   write
 general        1.166    0.4437  0.5142 0.02141
 vocation       1.164    0.4764  0.5956 0.02222
 
 Residual Deviance: 360 
 AIC: 376
\end{verbatim}

\end{frame}
%================================================================================================%
%\begin{frame}[fragile]
%
%\frametitle{Multinomial Regression with \texttt{R}}
%\Large
%\textbf{Test Statistics for computing $p-$values}
%\begin{verbatim}
%z <- summary(test)$coefficients/summary(test)$standard.errors
%z
% 
%          (Intercept) sesmiddle seshigh  write
% general        2.445   -1.2018  -2.261 -2.706
% vocation       4.485    0.6117  -1.650 -5.113
% \end{verbatim}
%
%\end{frame}
%=========================================%
\begin{frame}
\frametitle{Multinomial Logistic Regression with \texttt{R}}
\Large
\textbf{Wald Test}

\begin{itemize}
\item The Wald test in the context of logistic regression is used to determine whether a certain predictor variable X  is significant or not. 
\item It rejects the null hypothesis of the corresponding coefficient being zero. 

\item The test consists of dividing the value of the coefficient by standard error 
\end{itemize}




\end{frame}
%========================================================%
%================================================================================================%
\begin{frame}[fragile]

\frametitle{Multinomial Logistic Regression with \texttt{R}}
\large

\begin{verbatim}
# Coefficients Divided by Standard Errors
# Then Compute p-values.
# 2-tailed z test
# p.values
 
          (Intercept) sesmiddle seshigh     write
 general    1.448e-02    0.2294 0.02374 6.819e-03
 vocation   7.299e-06    0.5408 0.09895 3.176e-07
\end{verbatim}

\end{frame}

%================================================================================================%
\begin{frame}[fragile]

\frametitle{Multinomial Regression with \texttt{R}}
\Large
\begin{itemize}
\item Remark: Some output is generated by running the model, even though we are assigning the model to a new \texttt{R} object.
\item This model-running output includes some iteration history and includes the final \textbf{\textit{negative 
log-likelihood}} (+ 179.981726). 
\item This value multiplied by two is then seen in the model summary as the \textbf{\textit{Residual Deviance}} 
and it can be used in comparisons of nested models (360).
\end{itemize}
\end{frame}

%================================================================================================%
\begin{frame}[fragile]
	
	\frametitle{Multinomial Regression with \texttt{R}}
	\Large
	\begin{itemize}




\item As with many summary outputs, the output contains a column of coefficients and a column of standard errors. 
\item Each of these blocks has one row of values corresponding to a model equation. 
\item Focusing on the block of coefficients, we can look at the first row comparing \texttt{prog = "general"} to our baseline \texttt{prog = "academic"} and the second row comparing \texttt{prog = "vocation"} to our baseline \texttt{prog = "academic"}. 
\end{itemize}
\end{frame}
%================================================================================================%

\begin{frame}[fragile]
\frametitle{Multinomial Regression with \texttt{R}}
\Large
\begin{itemize}
\item If we consider our coefficients from the first row to be $b_1$ and our coefficients from the second row to be $b_2$, we can write our model equations:
\end{itemize}
{\normalsize
	
	\hspace{-0.95cm} \[ ln\left(\frac{P(prog=gen.)}{P(prog=acad.)}\right) = b_{10} + b_{11}(ses=2) + b_{12}(ses=3) + b_{13}write \] \[ ln\left(\frac{P(prog=voc.)}{P(prog=acad.)}\right) = b_{20} + b_{21}(ses=2) + b_{22}(ses=3) + b_{23}write \] 
}

\end{frame}
%================================================================================================%
\begin{frame}[fragile]
	
	\frametitle{Multinomial Regression with \texttt{R}}
	\Large
	\begin{itemize}

\item A one-unit increase in the variable \textbf{\textit{write}} is associated with the decrease in the log odds of being in general program vs. academic program in the amount of \textit{0.058 (b$_{13}$)}.
\smallskip
\item A one-unit increase in the variable \textbf{\textit{write}} is associated with the decrease 
in the log odds of being in vocation program vs. academic program in the amount of \textit{0.1136 (b$_{23}$)}.
\end{itemize}
\end{frame}
%================================================================================================%
\begin{frame}[fragile]

\frametitle{Multinomial Regression with \texttt{R}}
\Large
\begin{itemize}
\item The log odds of being in general program vs. in academic program will decrease by 
1.163 if moving from \textit{ses="low"} to \textit{ses="high"b$_{12}$}.
\smallskip
\item The log odds of being in general program vs. in academic 
program will decrease by 0.533 if moving from \textit{ses="low"} to \textit{ses="middle"b$_{11}$}, 
although this coefficient is not significant.
\end{itemize}
\end{frame}
%================================================================================================%
\begin{frame}[fragile]
	
	\frametitle{Multinomial Regression with \texttt{R}}
	\Large
	\begin{itemize}
\item The log odds of being in vocation program vs. in academic program will decrease by 
0.983 if moving from ses="low" to ses="high"(b$_{22}$).
\item The log odds of being in vocation program vs. in academic program 
will increase by 0.291 if moving from ses="low" to ses="middle"(b$_{21}$), although this coefficient is not signficant.
\end{itemize}
\end{frame}
\end{document}
%================================================================================================%
